\documentclass[11pt]{amsart}


\newcommand{\R}{\mathbb R}
\newcommand{\Z}{\mathbb Z}
\newcommand{\la}{\langle}
\newcommand{\ra}{\rangle}

\newtheorem{theorem}{Theorem}
\newtheorem{definition}[theorem]{Definition}
\newtheorem{corollary}[theorem]{Corollary}
\newtheorem{remark}[theorem]{Remark}
\newtheorem{example}[theorem]{Example}

\begin{document}
\title{Cartan matrix of a root system}
\author{Your name here}
\date{\today}
\maketitle

%\thispagestyle{empty}

{\Large  18.099 - 18.06 CI.} 

{Due on Monday, May 10 in class.} 

\vspace{1cm} 

{\it Write a paper proving the statements and working through the examples 
formulated below. Add your own 
examples, asides and discussions whenever needed. }

Let $V$ be a Euclidean space, that is  
a finite dimensional real linear space with a symmetric 
positive definite inner product $\la, \ra$. 

Recall that for a root system $\Delta$ in  
$V$, there exists a simple root system 
$\Pi \subset \Delta$ (not unique), such that  \begin{enumerate}
\item{ $\Pi$ is a basis in $V$;}
\item{Each root $\beta \in \Delta$ can be written as a linear 
combination of elements of $\Pi$ with integer coefficients of the same sign.}
\end{enumerate} 
 The root $\beta$ is 
positive if the coefficients are nonnegative, and negative otherwise. 
The set of all positive roots (the positive root system) 
associated to $\Pi$ is denoted $\Delta^+$. 

Below we will assume that the root system $\Delta$ is reduced (that is, 
for any $\alpha \in \Delta$, $2\alpha \notin \Delta$). 

We shall associate a ``Cartan matrix'' to the system $\Pi \subset \Delta$ 
and derive some properties of this matrix. An abstract Cartan matrix will 
be any square matrix with this list of properties. 
It turns out that 
an abstract Cartan matrix essentially determines the root system. 
In this paper we will work toward making this 
statement precise and proving it. The problem of classification of 
root systems is reduced then to the classification of the Cartan matrices. 

\begin{definition} Let $\Pi \subset \Delta$ be a chosen set of simple roots 
in 
the root system $\Delta$, and suppose that the simple roots are enumerated 
by $\{ \alpha_1, \alpha_2, \ldots, \alpha_l \}$, where $l = \dim(V)$. 
The \emph{Cartan matrix} of $(\Pi, \Delta)$ is the $l\times l$ matrix $A$ with 
$$ A_{ij} = \frac{2\langle \alpha_i, \alpha_j \rangle}{\langle \alpha_i, 
\alpha_i \rangle}.$$ 
\end{definition}

The Cartan matrix clearly depends on the enumeration of $\Pi$, but this 
dependence can be easily sorted out. Recall that a permutaion matrix $P^{ij}$ 
is a square matrix with $1 \leq i \neq j \leq l$, which is obtained from 
the identity matrix by exchanging rows $i$ and $j$. Note that a permutation 
matrix is always nonsingular. 
We need the following easy fact from linear algebra:

\begin{theorem} For any $l\times l$ matrix $B$, $B' = P^{ij} B (P^{ij})^{-1}$ 
is the matrix $B$ with $i$-th and $j$-th rows and columns exchanged. 
\end{theorem}  

\begin{corollary} Let $A$ be the Cartan matrix associated to $(\Pi, \Delta)$ 
with a fixed enumeration of the elements of $\Pi$, and let $A'$ be the 
Cartan matrix of the same root system with $\alpha_i$ and $\alpha_j$ 
exchanged in the enumeration of the simple roots. Then 
$A' = P^{ij} A (P^{ij})^{-1}$. 
\end{corollary}

\begin{definition} Two Cartan matrices are \emph{isomorphic} if they are 
conjugate by a product of permutation matrices. 
\end{definition} 

To understand the properties of $A$, we start with some examples. 

\begin{example} Let $(\Pi, \Delta)$ be the root system of type $A_n$ 
(see Example 10, \cite{2})
with the simple roots enumerated as listed there. Find the Cartan matrix 
of this system. 
\end{example} 

\begin{example} Let $\Delta$ be a root system of the type $B_4$ 
with the root vectors $\{ \pm e_i \pm e_j\}_{i \neq j} \cup \{ \pm e_i \}$, 
where $\{e_i\}_{i=1}^4$ is an orthonormal basis in $\R^n$. Check that 
  $\Pi = \{ e_1 - e_2, e_2 - e_3, e_3 -e_4, e_4 \}$ is 
a set of simple roots, and $\Delta^+ = \{ e_i \pm e_j\}_{i<j}\cup \{e_i\}$ 
- the associated set of positive roots in $\Delta$. 
Find the Cartan matrix of this root system.    
\end{example} 

We summarize the observed properties in the following statement. 
Recall that a square 
matrix $B$ is symmetric if $B_{ij}=B_{ji}$ for all $i,j$. A symmetric 
matrix $C$ is positive definite if $\langle C \cdot x, x \rangle >0$ for 
any nonzero $x \in V$. 

\begin{theorem} \label{A}
The Cartan matrix $A$ of a root system $(\Pi, \Delta)$ has 
the following properties: 
\begin{enumerate} 
\item{all entries $A_{ij}$ are integers;}
\item{$A_{ii}=2$ for all $i$;}
\item{$A_{ij} \leq 0$ for all $i \neq j$;}
\item{$A_{ij}=0$ if and only if $A_{ji}=0$;}
\item{there exists a diagonal matrix $D$ with positive entries such 
that $DAD^{-1}$ is symmetric positive definite.}
\end{enumerate}
\end{theorem} 
Hint: The only nontrivial statement is the last one. Try 
$D= diag(|\alpha_1|, \ldots |\alpha_l| )$, where $|\alpha_i|= 
\langle \alpha_i, \alpha_i \rangle^{1/2}$ is the length of a simple root.

\begin{example} Recall that two root systems in the same vector space 
are \emph{isomorphic} if they can be mapped to each other by a linear  
transformation preserving angles between the 
roots and relative lengths within each irreducible component. 
Classify the reduced root systems in $V= \R^2$ 
(follow the method indicated at the end of \cite{1}). Choose a 
simple root system in each case and find the Cartan matrix. Whenever 
necessary, conjugate by a diagonal matrix $D$ and check that the result 
is symmetric positive definite. Note that $DAD^{-1}$ may have non-integer 
entries. Is the matrix $D$ determined uniquely by the condition (5) in 
Theorem \ref{A}?
\end{example} 

Recall from \cite{1} that a root system is \emph{irreducible} if it 
doesn't admit a decomposition into two root systems 
$\Delta = \Delta' \cup \Delta''$, where each element of $\Delta' $ 
is orthogonal to each element of $\Delta''$. If $\Delta = \Delta' \cup 
\Delta''$ and $\Delta'$ is irreducible, then it is an irreducible component 
of $\Delta$. A root system is said to be reducible if it is not irreducible. 
Irreducibility of a root system can be detected on the level of Cartan 
matrices. 


\begin{theorem} A reduced root system is reducible if and only if 
for some choice of a simple root system and some enumeration of indices, 
the Cartan matrix is block diagonal with more than one block. 
\end{theorem} 
Hint:  The nontrivial part is to show that a root 
system corresponding to a block diagonal Cartan matrix $A$ is reducible. 
Suppose that $A$ has two blocks corresponding to the simple roots 
$\{\alpha_1, \ldots, \alpha_r \}$ and $\{\alpha_{r+1}, \ldots, \alpha_l\}$ 
respectively.  
For any $\alpha =\sum_{i=1}^l n_i \alpha_i \in \Delta $ you need to 
show that $\alpha$ belongs to one of the irreducible components of $\Delta$ 
spanned by the simple 
roots from one of the blocks. Assume that $\alpha$ is positive and proceed 
by induction in the number $\sum_{i=1}^l n_i$. You might need the property 
of root systems given in Theorem 10(2) in \cite{1}. 

A Cartan matrix such that any isomorphic Cartan
matrix has a single block
is said to be irreducible. 
Now we are ready to answer the question of uniqueness of $D$ in general. 
\begin{theorem} \label{D} 
The matrix $D$ is determined uniquely up to a scalar 
multiple on each block of $A$.
\end{theorem} 

Hint: Straightforward linear algebra. 
Suppose there are two diagonal matrices $D$ and $D'$ which 
symmetrize a Cartan matrix $A$, and compare the entries. 

Note that by Theorem \ref{A} (5), 
in the case that $A$ is irreducible, the matrix 
$D$ gives   
the relative lengths of the simple roots. Then the matrix elements 
of $A$ determine the relative angles between any two simple roots. We deduce 
the following 

\begin{corollary} The Cartan matrix for a set of simple roots determines 
that set of simple roots 
uniquely up to a root system isomorphism on $V$;
that is, up to a scalar multiple of an orthogonal transformation 
on each irreducible component.
\end{corollary} 

\begin{example} Let $\Delta = \Delta' \cup \Delta''$ be the union of 
the root systems $\Delta'$ of type $A_1$ and $\Delta''$ of type $B_2$.
(See Example 5 above for $A_n$ and Example 3 in \cite{3} for $B_2$). 
Find the Cartan matrix of this root system, and use a diagonal 
matrix $D$ to symmetrize it. 
\end{example} 


\begin{thebibliography}{2}

\bibitem[1]{1} Your classmate, {\it Abstract root systems}, 
preprint, MIT, 2004. 

\bibitem[2]{2} Your classmate, {\it Simple and positive roots},
preprint, MIT, 2004.

\bibitem[3]{3} Your classmate, {\it Properties of simple roots},
preprint, MIT, 2004.

\end{thebibliography}


\end{document}

























