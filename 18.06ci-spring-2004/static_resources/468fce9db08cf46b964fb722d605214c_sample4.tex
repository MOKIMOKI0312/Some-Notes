\documentclass[11pt]{amsart}
\usepackage{amssymb,amsmath}

\newcommand{\R}{\mathbb R}
\newcommand{\C}{\mathbb C}
\newtheorem{theorem}{Theorem}
\newtheorem{lemma}[theorem]{Lemma}

\begin{document}


\begin{theorem} 
The field $\C$ is a vector space over $\R$
\end{theorem}



\begin{theorem}
A linear map is injective if and only if its kernel is zero.
\end{theorem}

\begin{proof} 
The proof uses Lemma \ref{nolemma}
\end{proof}


\begin{lemma} \label{nolemma}
Cannot think of one for such a trivial theorem. 
\end{lemma}


\end{document}

This sample contains a new environment: theorem. You can use 
it in the text, putting your text between \begin{theorem} 
and \end{theorem}. LaTeX counts theorems for you. 

Note that it is convenient to use packages provided by the 
American Mathematical Society to produce nice math symbols. 
The packages are available in Athena. 

You can define ``macros'' for frequently used commands, as {\mathbb R} 
above. The { } brackets are used to separate the field with the 
name and the contents of the new command.  

This example also uses cross references. You can put \label inside 
any environment and \ref whenever you want to refer to it. This 
will produce the number of the environment. So you don't have 
to keep track of the counting of your theorems etc. 
