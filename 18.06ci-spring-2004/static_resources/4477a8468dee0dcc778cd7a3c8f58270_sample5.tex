\documentclass[11pt]{amsart}
\usepackage{amssymb,amsmath}

\newcommand{\R}{\mathbb R}
\newcommand{\C}{\mathbb C}
\newtheorem{theorem}{Theorem}
\newtheorem{lemma}[theorem]{Lemma}


\begin{document}


\begin{theorem} 
The field $\C$ is a vector space over $\R$.
\end{theorem}

\begin{theorem}
A matrix of a change of the basis for a finite dimensional 
linear space is nondegenerate.  
\end{theorem}

\begin{proof} 

Consider a linear map 
$$f : M \to M$$  
such that  
\begin{equation} \label{map}
 f(e_i) = g_i. 
\end{equation} 
In (\ref{map}), $\{e_i\}_{i=1}^n$ and $\{ g_i \}_{i=1}^n$ are two bases 
in $M$. 

The rest of the proof can be found, e.g.,  
in \cite{S}\footnote{But this is not a 
way to solve a homework problem}. 
\end{proof}

\begin{thebibliography}{2}

\bibitem[S]{S} G. Strang, {\it Introduction to linear algebra}, third ed, 
Wellesley-Cambridge press, 2003

\end{thebibliography}

\end{document}


This is an example of a bibliography. You can have references to books and 
papers in the text and list them at the end. 

Important: to get the references right, run latex twice. 

Note the difference between the two ways to display math:
$$             $$ makes a line with your equation;
\begin{equation}       \end{equation} makes a line with your equation 
and gives it a number, so later you can refer to it in the text, using 
\label{...}  and \ref{...} operators.  
